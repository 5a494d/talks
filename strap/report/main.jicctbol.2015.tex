% This is "sig-alternate.tex" V2.0 May 2012
% This file should be compiled with V2.5 of "sig-alternate.cls" May 2012
%
% This example file demonstrates the use of the 'sig-alternate.cls'
% V2.5 LaTeX2e document class file. It is for those submitting
% articles to ACM Conference Proceedings WHO DO NOT WISH TO
% STRICTLY ADHERE TO THE SIGS (PUBS-BOARD-ENDORSED) STYLE.
% The 'sig-alternate.cls' file will produce a similar-looking,
% albeit, 'tighter' paper resulting in, invariably, fewer pages.
%
% ----------------------------------------------------------------------------------------------------------------
% This .tex file (and associated .cls V2.5) produces:
%       1) The Permission Statement
%       2) The Conference (location) Info information
%       3) The Copyright Line with ACM data
%       4) NO page numbers
%
% as against the acm_proc_article-sp.cls file which
% DOES NOT produce 1) thru' 3) above.
%
% Using 'sig-alternate.cls' you have control, however, from within
% the source .tex file, over both the CopyrightYear
% (defaulted to 200X) and the ACM Copyright Data
% (defaulted to X-XXXXX-XX-X/XX/XX).
% e.g.
% \CopyrightYear{2007} will cause 2007 to appear in the copyright line.
% \crdata{0-12345-67-8/90/12} will cause 0-12345-67-8/90/12 to appear in the copyright line.
%
% ---------------------------------------------------------------------------------------------------------------
% This .tex source is an example which *does* use
% the .bib file (from which the .bbl file % is produced).
% REMEMBER HOWEVER: After having produced the .bbl file,
% and prior to final submission, you *NEED* to 'insert'
% your .bbl file into your source .tex file so as to provide
% ONE 'self-contained' source file.
%
% ================= IF YOU HAVE QUESTIONS =======================
% Questions regarding the SIGS styles, SIGS policies and
% procedures, Conferences etc. should be sent to
% Adrienne Griscti (griscti@acm.org)
%
% Technical questions _only_ to
% Gerald Murray (murray@hq.acm.org)
% ===============================================================
%
% For tracking purposes - this is V2.0 - May 2012

\documentclass{sig-alternate}
\begin{document}

% --- Author Metadata here ---

%\conferenceinfo{WOODSTOCK}{'97 El Paso, Texas USA}
%\ CopyrightYear{2007} % Allows default copyright year (20XX) to be 
%\ over-ridden - IF NEED BE.
%\ crdata{0-12345-67-8/90/01}  % Allows default copyright data (0-89791-88-6/97/05) to be over-ridden - IF NEED BE.

% --- End of Author Metadata ---

\title{STRAP,  \titlenote{( Basado en el Sistema de autentificaci\'on - Centinela ) }}

%\subtitle{[Extended Abstract]
%	\titlenote{A full version of this paper is available as
%		\textit{Author's Guide to Preparing ACM SIG Proceedings Using
%		\LaTeX$2_\epsilon$\ and BibTeX} at
%		\texttt{www.acm.org/eaddress.htm}
%	}
%}

%
% You need the command \numberofauthors to handle the 'placement
% and alignment' of the authors beneath the title.
%
% For aesthetic reasons, we recommend 'three authors at a time'
% i.e. three 'name/affiliation blocks' be placed beneath the title.
%
% NOTE: You are NOT restricted in how many 'rows' of
% "name/affiliations" may appear. We just ask that you restrict
% the number of 'columns' to three.
%
% Because of the available 'opening page real-estate'
% we ask you to refrain from putting more than six authors
% (two rows with three columns) beneath the article title.
% More than six makes the first-page appear very cluttered indeed.
%
% Use the \alignauthor commands to handle the names
% and affiliations for an 'aesthetic maximum' of six authors.
% Add names, affiliations, addresses for
% the seventh etc. author(s) as the argument for the
% \additionalauthors command.
% These 'additional authors' will be output/set for you
% without further effort on your part as the last section in
% the body of your article BEFORE References or any Appendices.

\numberofauthors{8} 


% of EIGHT authors. SIX appear on the 'first-page' (for formatting
% reasons) and the remaining two appear in the \additionalauthors section.
%

\author{
% You can go ahead and credit any number of authors here,
% e.g. one 'row of three' or two rows (consisting of one row of three
% and a second row of one, two or three).
%
% The command \alignauthor (no curly braces needed) should
% precede each author name, affiliation/snail-mail address and
% e-mail address. Additionally, tag each line of
% affiliation/address with \affaddr, and tag the
% e-mail address with \email.
%
% 1st. author
\alignauthor
Lipa Challapa \\ Jorge Rubens\\
       \affaddr{Universidad Mayor de \\San Sim\'on}\\
       \affaddr{Calle Sucre y parque La Torre}\\
       \affaddr{Cochabamba, Bolivia}\\
       \email{jorge.lipa@gmail.com}\\
}

% There's nothing stopping you putting the seventh, eighth, etc.
% author on the opening page (as the 'third row') but we ask,
% for aesthetic reasons that you place these 'additional authors'
% in the \additional authors block, viz.

%\additionalauthors{Additional authors: John Smith, 
%	email: \texttt{jsmith@affiliation.org} and Julius Kumquat
%	The Kumquat Consortium, email: \texttt{jpkumquat@consortium.net}.
%	}

%\date{21}

% Just remember to make sure that the TOTAL number of authors
% is the number that will appear on the first page PLUS the
% number that will appear in the \additionalauthors section.

\maketitle

\begin{abstract}

STRAP proveera seguimiento a objetos identificados con RFID, para el 
analisis de datos, mediante el cual facilitar la generacion de reportes 
de forma automatizada.

Los dispositivos de recoleccion de datos que se utilizara son telefonos 
moviles con microcontroladores RFID integrados, llamados tambien NFC. 
Los datos seran enviados por los diferentes dispositivos por servicios 
disponibles en internet o intranet. 

Este sistema puede ser instalado en direntes sistemas que pueden ser 
utilizados para diferentes monitoreos en tiempo real.
\end{abstract}

% A category with the (minimum) three required fields

\category{B.4.1}{ Input/Output and Data Communications}{Data Communications Devices}

%A category including the fourth, optional field follows...

\category{H.2.4 }{Systems}{Concurrency}

\category{H.3.5 }{Online Information Services}{Commercial services}

%\terms{}

\keywords{Arduino,nfc,rfid,services,java,scala}

\section{Introducci\'on}
A menudo, el seguimiento y recoleccion de informacion llega a ser muy 
repetitivos o complicado en este caso por la complejidad de los datos. 
En una digital, la recoleccion de informacion con dispositivos de control 
como el codigo de barras y QR llegan a tener problemas, ya sea por el 
desgaste fisico o la lectura de este.

Con la tecnologia RFID esto puede realizarse mas rapido, gracias a los 
dispositivos de recoleccion que se pueden ser de largo o corto alcance. 
Una de las tecnologias que se utilizara en este proyecto son los de corto 
alcance que trabajan a una frecuencia de 12..5 kHz y 13.5 MHz que vienen 
integrados en algunos telefonos moviles con el nombre de NFC.

\section{Antecedentes}

RFID, \( Radio Frequency IDentification \), Identificaci\'on por Radio Frecuencia esta orientado a la identificacion \'unica de objetos mediante etiquetas que transmiten informacion mediante ondas de radio ajustadas a 
una frecuencia especifica.\\

Con el intercambio de informacion digital, se puede identificar el objeto 
rapidamente, obteniendo informaci\'on espec\'ifica y agregarla a un servicio 
para gestionarla e interpretarla.\\

Algun tiempo atras, la recoleccion se la realizaba a mano o utilizando algun 
tipo de identificacion grafica que podia ser leida por una persona, los 
primeros pasos para la identificacion autom\'atica llegar con los codigos 
de barra y posteriormente codigos QR, estos ultimos podian ser leidos por 
dispositivo lectores, el inconventiente con estos dispositivos, era que 
un operador era necesario en la mayoria de los casos.\\

Hoy en dia solo algunos paises implementan RFID en el uso cotidiano, ya sea 
para servicios basicos, en el sector empresarial se implementa para 
seguimiento y control de productos. En otros sectores para geolocalizaci\'on 
, dentro de algunos anios llegara a ser mas utilizada en sectores 
productivos como personales.\\

\section{Justificaci\'on}




\section{Planteamiento del problema}

\section{Objetivos}

\subsection{Objetivo general}

list

\subsection{Objetivos espec\'ificos}

\section{Hipotesis}

\section{Novedad y aporte tecnol\'ogico}

\section{Dise\~no metodol\'ogico y te\'orico}

\section{Desarrollo del proyecto}

\section{Desarrollo del proyecto}


\section{Conclusiones y recomendaciones}

%\end{document}  % This is where a 'short' article might terminate

%ACKNOWLEDGMENTS are optional

\section{Agradecimientos}

%
% The following two commands are all you need in the
% initial runs of your .tex file to
% produce the bibliography for the citations in your paper.

\bibliographystyle{abbrv}
\bibliography{sigproc}  % sigproc.bib is the name of the Bibliography in this case

% You must have a proper ".bib" file
%  and remember to run:
% latex bibtex latex latex
% to resolve all references
%
% ACM needs 'a single self-contained file'!
%
%APPENDICES are optional
%\balancecolumns

%\appendix

%Appendix A

%\section{Headings in Appendices}

%Appendix B

%\section{More Help for the Hardy}

%\balancecolumns % GM June 2007
% That's all folks!

\end{document}
